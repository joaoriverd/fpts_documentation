\documentclass[fleqn]{article}
\usepackage[english]{babel}
\usepackage{a4wide}
\usepackage{latexsym}
\usepackage{times}
%\usepackage{theorem}
\usepackage{url}
\usepackage[final]{graphics}
\usepackage{amsmath,amssymb}
\usepackage{amsfonts}
\usepackage{array}
\usepackage{calc}
\usepackage{xspace}
\usepackage{color}
\usepackage{epsfig}
\usepackage{subfigure}
\usepackage{float}
\usepackage{stmaryrd}
\usepackage{color}
\usepackage{mathtools}
\usepackage{graphicx}
\usepackage{epstopdf}
\usepackage{listings}
\usepackage{color}
\usepackage{booktabs,caption}
%\usepackage[flushleft]{threeparttable}

\DeclareMathAlphabet{\mathpzc}{OT1}{pzc}{m}{it}
\usepackage{amsthm,amssymb}

\usepackage{amssymb}
\usepackage{graphicx}
\usepackage{epstopdf}
\usepackage{mathtools}

\usepackage{algorithm}
\usepackage[noend]{algpseudocode}

\usepackage{fouriernc}
\pagestyle{plain}
\usepackage{float}
\usepackage[hidelinks]{hyperref}

\usepackage{array}
\newcolumntype{L}[1]{>{\raggedright\let\newline\\\arraybackslash\hspace{0pt}}m{#1}}
\newcolumntype{C}[1]{>{\centering\let\newline\\\arraybackslash\hspace{0pt}}m{#1}}
\newcolumntype{R}[1]{>{\raggedleft\let\newline\\\arraybackslash\hspace{0pt}}m{#1}}

\newtheorem{theorem}{Theorem}
\newtheorem{fact}{Fact}
\newtheorem{hypothesis}{Hypothesis}
\newtheorem{lemma}{Lemma}
\newtheorem{definition}{Definition}

\definecolor{codegreen}{rgb}{0,0.6,0}
\definecolor{codegray}{rgb}{0.5,0.5,0.5}
\definecolor{codepurple}{rgb}{0.58,0,0.82}
\definecolor{backcolour}{rgb}{0.95,0.95,0.92}

\lstdefinestyle{mystyle}{
	backgroundcolor=\color{backcolour},   
	commentstyle=\color{codegreen},
	keywordstyle=\color{magenta},
	numberstyle=\tiny\color{codegray},
	stringstyle=\color{codepurple},
	basicstyle=\footnotesize,
	breakatwhitespace=false,         
	breaklines=true,                 
	captionpos=b,                    
	keepspaces=true,                 
	numbers=left,                    
	numbersep=5pt,                  
	showspaces=false,                
	showstringspaces=false,
	showtabs=false,                  
	tabsize=2
}

\lstset{style=mystyle}


\usepackage{fouriernc}
\pagestyle{plain}

%% macros.tex

\title{\sf Proofs}
\author{{\sf H.J. Rivera Verduzco 0977393}\\
{\footnotesize\sl P.O.~Box 513, 5600 MB Eindhoven, The Netherlands}\\
{\footnotesize \sl Email: \tt H.J.Rivera.Verduzco@student.tue.nl}}
%\date{}
\begin{document}
\maketitle

%\begin{abstract}
%\noindent
% Add abstract here %
%\end{abstract}

\section{Blocking tasks}

%\begin{lemma}
%	Given a schedule for a task-set $\mathpzc{T}$ of $n$ independent tasks with an \textit{occupied interval} $OI^{\mathpzc{T}}_n (k)$ for job $k$ of task $\tau_n$. If for a new task-set $\mathpzc{T} \cup \tau_{n+1}$ a job $\iota_{n+1, k\prime}$ is scheduled within an interval $[t_1,t_2)$, where there is an amount of time $C_{n+1}$ available for $\tau_{n+1}$ and $t_2$ is equal to the start time of $OI^{\mathpzc{T}}_n (k)$, then the schedule of all jobs within $OI^{\mathpzc{T}}_n (k)$ remains the same after introducing $\tau_{n+1}$.
%\end{lemma}

\begin{lemma}
	Given a task-set $\mathpzc{T}$ of $n$ independent tasks scheduled using FPTS, and a level-$n$ busy period $[t_s,t_f]$ of such a schedule. If for a new task-set $\mathpzc{T} \cup \{\tau_{n+1}\}$, the phase of $\tau_{n+1}$ is selected in such a way that a job $k$ of $\tau_{n+1}$ is scheduled in an interval $[t_1,t_s)$ where there is an amount of time $C_{n+1}$ available for the execution of $\tau_{n+1}$, and job $k$ does not experience blocking by previous jobs of $\tau_{n+1}$, then the schedule of all jobs in $[t_s,t_f)$ remains the same after introducing $\tau_{n+1}$.
\end{lemma}

%\begin{proof}
%	Assume that task $\tau_{n+1}$ is scheduled in such a way that a job $\iota_{n+1, k\prime}$ occurs within the interval $[t_1,t_2)$ as described in Lemma 1. We now show that, under this condition, the schedule in interval $OI^{\mathpzc{T}}_n (k)$ remains the same.
%	
%	After introducing $\tau_{n+1}$, the schedule in $OI^{\mathpzc{T}}_n (k)$ could be affected by the job $\iota_{n+1, k\prime}$, by any previous job $\iota_{n+1, k_1}$ with $k_1 < k\prime$, or by any further job $\iota_{n+1, k_2}$ with $k_2 > k\prime$. Since we assumed that job $\iota_{n+1, k\prime}$ occurs in $[t_1,t_2)$ and there is enough available time in this interval for its execution, it holds that $f_{n+1,k\prime} \leq t_2$; hence, this job does not affect the schedule in $OI^{\mathpzc{T}}_n (k)$. Furthermore, any previous job $\iota_{n+1, k_1}$ should has finished before $\iota_{n+1, k \prime}$ starts; hence, it will not affect the schedule in $OI^{\mathpzc{T}}_n (k)$ either. Finally, any job $\iota_{n+1, k_2}$ should start after $f_{n+1,k\prime}$; however it cannot start in the interval $[f_{n+1,k\prime},t_2) \cup OI^{\mathpzc{T}}_n (k)$ because there is no enough time available for $\tau_{n+1}$. Therefore, the start time of any job $\iota_{n+1, k_2}$ will occur after $OI^{\mathpzc{T}}_n (k)$, and, as a result, it will not affect the schedule in $OI^{\mathpzc{T}}_n (k)$.
%\end{proof}

\begin{proof}
	Assume a level-$n$ busy period $[t_s,t_f]$ for task-set $\mathpzc{T}$, and also assume that task $\tau_{n+1}$ is scheduled in such a way that a job $k$ of $\tau_{n+1}$ occurs in the interval $[t_1,t_s)$ as described in Lemma 1. We now show that, under this condition, the schedule in the interval $[t_s,t_f)$ remains the same after introducing $\tau_{n+1}$.
	
	After introducing $\tau_{n+1}$, the schedule in $[t_s,t_f)$ can be affected by any blocking introduced by the job $k$ of $\tau_{n+1}$, or by any later job $k_{post}$ of $\tau_{n+1}$ with $k_{post} > k$. Recall that we assumed that job $k$ of $\tau_{n+1}$ occurs in $[t_1,t_s)$, and it does not experience blocking by previous jobs; furthermore, there is enough time available in this interval for its execution. Hence, this job can finalize before $t_s$ without inducing blocking to the jobs in $[t_s,t_f)$, and therefore it will not affect the schedule in $[t_s,t_f)$. Finally, any job $k_{post}$ of $\tau_{n+1}$ should start after job $k$; however, it cannot start in the interval $[f_{n+1,k},t_f)$ because there is no idle time available for $\tau_{n+1}$. Therefore, the start time of any job $k_{post}$ of $\tau_{n+1}$ will be after or at time $t_f$. As a result, it will not affect the schedule in $[t_s,t_f)$.
\end{proof}

%\begin{lemma}
%	Given a schedule for a task-set $\mathpzc{T}$ of $n$ independent tasks with an \textit{occupied interval} $OI^{\mathpzc{T}}_n (k)$ for job $k$ of task $\tau_n$, it is always possible for a new task-set $\mathpzc{T} \cup \tau_{n+1}$ to schedule $\tau_{n+1}$ in such a way that a job $\iota_{n+1, k\prime}$ occurs within an interval $[t_1,t_2)$, where there is an amount of time $C_{n+1}$ available for $\tau_{n+1}$ and $t_2$ is equal to the start time of $OI^{\mathpzc{T}}_n (k)$.
%\end{lemma}

\begin{lemma}
	Given a task-set $\mathpzc{T}$ of $n$ independent tasks scheduled under FPTS, a level-$n$ busy period $[t_s,t_f]$ of such a schedule, and a new task-set $\mathpzc{T} \cup \{\tau_{n+1}\}$ where $U^{\mathpzc{T} \cup \tau_{n+1}}<1$ or $U^{\mathpzc{T} \cup \tau_{n+1}} = 1$ and the lcm of the periods exists. It is always possible for task-set $\mathpzc{T} \cup \{\tau_{n+1}\}$ to select the phase of $\tau_{n+1}$ in such a way that a job $k$ of $\tau_{n+1}$ is scheduled in an interval $[t_1,t_s)$ where there is an amount of time $C_{n+1}$ available for the execution of $\tau_{n+1}$, and job $k$ does not experience blocking by previous jobs of $\tau_{n+1}$.
\end{lemma}

\begin{proof}
	Since $U^{\mathpzc{T} \cup \tau_{n+1}}<1$ or $U^{\mathpzc{T} \cup \tau_{n+1}} = 1$ and the lcm of the periods exists, it is always possible to find a contingent interval $[t_1,t_s)$ before the level-$n$ busy period $[t_s,t_f]$ where there is an amount of time $C_{n+1}$ available for the execution of $\tau_{n+1}$. The only reason for which a job $k$ of $\tau_{n+1}$ may not be able to execute in $[t_1,t_s)$ is that it could experience some blocking from previous jobs delaying the start time of such a job $k$. However, it is possible to change the phase of $\phi_{n+1}$ in order to move the activations of the jobs of $\tau_{n+1}$ at an earlier moment in time, and therefore allow enough space before $[t_1,t_s)$ for the execution of the earlier jobs of $\tau_{n+1}$ before job $k$ that may delay its start time. Since the number of jobs of $\tau_{n+1}$ that may delay the start time of job $k$ is bounded by $wl_{n+1}-1$, it is always possible to find enough space before $[t_1,t_s)$ for their execution.
\end{proof}

%\begin{proof}
%	Since $U^{\mathpzc{T} \cup \tau_{i+1}}<1$ or $U^{\mathpzc{T} \cup \tau_{i+1}} = 1$ and the lcm of the periods exists, it is always possible to find a contingent interval $[t_i,t_2)$ before $OI^{\mathpzc{T}}_n (k)$ where there is an amount of time $C_{i+1}$ available for the execution of $\tau_{i+1}$. The only reason for which a job $\iota_{i+1, k\prime}$ of $\tau_{i+1}$ may not be able to execute within $[t_1,t_2)$ is that it could experience some blocking by previous jobs delaying the start time of $\iota_{i+1, k\prime}$. However, it is possible to change the phase $\phi_{i+1}$ in order to move the activations of the jobs of $\tau_{i+1}$ at an earlier moment in time, and therefore allow enough space before $[t_1,t_2)$ for the execution of the jobs previous to $\iota_{i+1, k\prime}$ that may delay its start time. Since the number of jobs of $\tau_{i+1}$ that may delay the start time of $\iota_{i+1, k\prime}$ is bounded by $wl_{i+1}-1$, it is always possible to find enough space before $[t_1,t_2)$ for their execution.
%\end{proof}

\begin{lemma}
	Let all tasks of a set $\mathpzc{T}$ be strictly periodic and let $U^{\mathpzc{T}}<1$ or $U^{\mathpzc{T}}=1$ and the lcm of the periods exists. Under FPTS, the best-case response time $BR_i$ of a task $\tau_i \in \mathpzc{T}$ is not influenced by its lower priority tasks.
\end{lemma}

\begin{proof}
	Assume that we find a schedule for a subset of task-set $\mathpzc{T}$ defined as $\mathpzc{T}_i = \{\tau_a | a \leq i, \tau_a \in \mathpzc{T}\}$ where the \textit{best-case response time} $BR_i$ of $\tau_i$ is assumed by job $k^{bcrt}$ in the level-$i$ busy period $[t_{s},t_{f}]$. Note that task-set $\mathpzc{T}_i$ contains the same tasks as $\mathpzc{T}$ without the lower priority tasks of $\tau_i$. Given lemmas 1 and 2, we can construct a schedule for a task-set $\mathpzc{T}_{i+1}$ where task $\tau_{i+1}$ is scheduled in such a way that a job $k$ occurs in a contingent time interval $[t_1, t_s)$ before $[t_s,t_f)$. Hence, the schedule in time interval $[t_s,t_f)$ will remain unchanged and the \textit{best-case response time} $BR_i$ of $\tau_i$ will remain the same in $\mathpzc{T}_{i+1}$. Furthermore, note that since job $k$ of $\tau_{i+1}$ occurs in a contingent time interval $[t_1, t_s)$ before $[t_s,t_f)$, these two time intervals are contained in the level-$\{i+1\}$ busy period $[t\prime_s,t\prime_f]$; hence, job $k^{bcrt}$ of $\tau_i$ occurs in $[t\prime_s,t\prime_f]$ as well. Based on this observation and using lemmas 1 and 2 again, we can find a schedule for a task-set $\mathpzc{T}_{i+2}$ where the jobs in the  level-$\{i+1\}$ busy period $[t\prime_s,t\prime_f]$ remain unchanged and, therefore, $BR_i$ is preserved. Since we can continue constructing schedules preserving the \textit{best-case response time} of $\tau_i$ until $\mathpzc{T}_{i+l} = \mathpzc{T}$, we conclude that the \textit{best-case response time} $BR_i$ of a task $\tau_i$ is not affected by its lower priority tasks.
\end{proof}

%\begin{proof}
%	Assume that we find a schedule for a subset of task-set $\mathpzc{T}$ defined as $\mathpzc{T}_i = \{\tau_a | a \leq i, \tau_a \in \mathpzc{T}\}$ where the \textit{best-case response time} of $\tau_i$ is assumed and it occurs in an occupied interval $OI^{\mathpzc{T}_i}_i(k^{bcrt})$. Note that task-set $\mathpzc{T}_i$ contains the same tasks as $\mathpzc{T}$ without the lower priority tasks of $\tau_i$. Given Lemma 1 and Lemma 2, we can construct a schedule for a task-set $\mathpzc{T}_{i+1}$ where task $\tau_{i+1}$ is scheduled in such a way that a job $\iota_{i+1, k\prime}$ occurs in a contingent time interval $[t_1, t_2)$ before $OI^{\mathpzc{T}_i}_i(k^{bcrt})$. Hence, the jobs in the $OI^{\mathpzc{T}_i}_i(k^{bcrt})$ interval will remain unchanged and the \textit{best-case response time} $BR_i$ of $\tau_i$ will remain the same in $\mathpzc{T}_{i+1}$. Furthermore, note that since $\iota_{i+1, k\prime}$ occurs in a contingent time interval before the interval $OI^{\mathpzc{T}_i}_i(k^{bcrt})$, these two time intervals are contained in the longer occupied interval $OI^{\mathpzc{T}_{i+1}}_{i+1}(k\prime)$; hence, $\iota_{i, k^{bcrt}}$ occurs in $OI^{\mathpzc{T}_{i+1}}_{i+1}(k\prime)$ as well. Based on this observation and using Lemma 1 and Lemma 2 again, we can find a schedule for a task-set $\mathpzc{T}_{i+2}$ where the jobs in $OI^{\mathpzc{T}_{i+1}}_{i+1}(k\prime)$ remain unchanged and, therefore, $BR_i$ is preserved. Since we can continue constructing schedules preserving the \textit{best-case response time} of $\tau_i$ until $\mathpzc{T}_{i+l} = \mathpzc{T}$, we conclude that the \textit{best-case response time} of a task $\tau_i$ is not affected by its lower priority tasks.
%\end{proof}

\section{Higher priority tasks}

\begin{lemma}
	Let $\mathpzc{T}$ be a set of strictly periodic tasks with a task $\tau_i \in \mathpzc{T}$. The threshold of higher priority tasks do not influence the response time of a job of $\tau_i$.
\end{lemma}

\begin{proof}
	In order to prove Lemma 4, first note that the response time $R_{i,k}$ of a job $\iota_{i,k}$ of $\tau_i$ is equal to $S_{i,k}+H_{i,k}$. Therefore, it is sufficient to show that the thresholds of higher priority tasks do not have an influence on the start time $S_{i,k}$ and the hold time $H_{i,k}$.
	
	\{\textit{Influence on} $S_{i,k}$\}. Under FPTS, a necessary condition for a job $\iota_{i,k}$ to start at a time $t$ after its activation is that the priority of $\tau_i$ is the highest among all tasks with pending load at time $t$. Since this condition only depends on the priority of the tasks, higher priority tasks will always affect the start time $S_{i,k}$ of $\iota_{i,k}$ independently of their thresholds.
	
	\{\textit{Influence on} $H_{i,k}$\}. Under FPTS, a job $\iota_{i,k}$ of a task $\tau_i$ can only be preempted by a higher priority task $\tau_h$ iff $\pi_h > \theta_i$. Since this condition is independent of the threshold $\theta_h$ of $\tau_h$, we conclude that the thresholds of higher priority tasks of $\tau_i$ do not have an influence on the hold time $H_{i,k}$ of $\iota_{i,k}$.
\end{proof}

\begin{thebibliography}{10}
	

\end{thebibliography}
\end{document}

