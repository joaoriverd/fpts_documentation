\documentclass[fleqn]{article}
\usepackage[english]{babel}
\usepackage{a4wide}
\usepackage{latexsym}
\usepackage{times}
\usepackage{theorem}
\usepackage{url}
\usepackage[final]{graphics}
\usepackage{amsmath,amssymb}
\usepackage{amsfonts}
\usepackage{array}
\usepackage{calc}
\usepackage{xspace}
\usepackage{color}
\usepackage{epsfig}
\usepackage{subfigure}
\usepackage{float}
\usepackage{stmaryrd}
\usepackage{color}
\usepackage{mathtools}
\usepackage{graphicx}
\usepackage{epstopdf}
\usepackage{listings}
\usepackage{color}
\usepackage{booktabs,caption}
%\usepackage[flushleft]{threeparttable}

\DeclareMathAlphabet{\mathpzc}{OT1}{pzc}{m}{it}
\usepackage{amsthm,amssymb}

\usepackage{amssymb}
\usepackage{graphicx}
\usepackage{epstopdf}
\usepackage{mathtools}

\usepackage{algorithm}
\usepackage[noend]{algpseudocode}

\usepackage{fouriernc}
\pagestyle{plain}
\usepackage{float}
\usepackage[hidelinks]{hyperref}

\usepackage{array}
\newcolumntype{L}[1]{>{\raggedright\let\newline\\\arraybackslash\hspace{0pt}}m{#1}}
\newcolumntype{C}[1]{>{\centering\let\newline\\\arraybackslash\hspace{0pt}}m{#1}}
\newcolumntype{R}[1]{>{\raggedleft\let\newline\\\arraybackslash\hspace{0pt}}m{#1}}

\newtheorem{theorem}{Theorem}
\newtheorem{fact}{Fact}
\newtheorem{hypothesis}{Hypothesis}
\newtheorem{lemma}{Lemma}
\newtheorem{definition}{Definition}

\definecolor{codegreen}{rgb}{0,0.6,0}
\definecolor{codegray}{rgb}{0.5,0.5,0.5}
\definecolor{codepurple}{rgb}{0.58,0,0.82}
\definecolor{backcolour}{rgb}{0.95,0.95,0.92}

\lstdefinestyle{mystyle}{
	backgroundcolor=\color{backcolour},   
	commentstyle=\color{codegreen},
	keywordstyle=\color{magenta},
	numberstyle=\tiny\color{codegray},
	stringstyle=\color{codepurple},
	basicstyle=\footnotesize,
	breakatwhitespace=false,         
	breaklines=true,                 
	captionpos=b,                    
	keepspaces=true,                 
	numbers=left,                    
	numbersep=5pt,                  
	showspaces=false,                
	showstringspaces=false,
	showtabs=false,                  
	tabsize=2
}

\lstset{style=mystyle}


\usepackage{fouriernc}
\pagestyle{plain}

%% macros.tex

\title{\sf Proofs}
\author{{\sf H.J. Rivera Verduzco 0977393}\\
{\footnotesize\sl P.O.~Box 513, 5600 MB Eindhoven, The Netherlands}\\
{\footnotesize \sl Email: \tt H.J.Rivera.Verduzco@student.tue.nl}}
%\date{}
\begin{document}
\maketitle

%\begin{abstract}
%\noindent
% Add abstract here %
%\end{abstract}

\section{Blocking tasks}

\begin{lemma}
	Given a schedule for a task-set $\mathpzc{T}$ of $n$ independent tasks with an \textit{occupied interval} $OI^{\mathpzc{T}}_n (k)$ for job $k$ of task $\tau_n$. If for a new task-set $\mathpzc{T} \cup \tau_{n+1}$ a job $\iota_{n+1, k\prime}$ is scheduled within an interval $[t_1,t_2)$, where there is an amount of time $C_{n+1}$ available for $\tau_{n+1}$ and $t_2$ is equal to the start time of $OI^{\mathpzc{T}}_n (k)$, then the schedule of all jobs within $OI^{\mathpzc{T}}_n (k)$ remains the same after introducing $\tau_{n+1}$.
\end{lemma}

\begin{proof}
	Assume that task $\tau_{n+1}$ is scheduled in such a way that a job $\iota_{n+1, k\prime}$ occurs within the interval $[t_1,t_2)$ as described in Lemma 1. We now show that, under this condition, the schedule in interval $OI^{\mathpzc{T}}_n (k)$ remains the same.
	
	After introducing $\tau_{n+1}$, the schedule in $OI^{\mathpzc{T}}_n (k)$ could be affected by the job $\iota_{n+1, k\prime}$, by any previous job $\iota_{n+1, k_1}$ with $k_1 < k\prime$, or by any further job $\iota_{n+1, k_2}$ with $k_2 > k\prime$. Since we assumed that job $\iota_{n+1, k\prime}$ occurs in $[t_1,t_2)$ and there is enough available time in this interval for its execution, it holds that $f_{n+1,k\prime} \leq t_2$; hence, this job does not affect the schedule in $OI^{\mathpzc{T}}_n (k)$. Furthermore, any previous job $\iota_{n+1, k_1}$ should has finished before $\iota_{n+1, k \prime}$ starts; hence, it will not affect the schedule in $OI^{\mathpzc{T}}_n (k)$ either. Finally, any job $\iota_{n+1, k_2}$ should start after $f_{n+1,k\prime}$; however it cannot start in the interval $[f_{n+1,k\prime},t_2) \cup OI^{\mathpzc{T}}_n (k)$ because there is no enough time available for $\tau_{n+1}$. Therefore, the start time of any job $\iota_{n+1, k_2}$ will occur after $OI^{\mathpzc{T}}_n (k)$, and, as a result, it will not affect the schedule in $OI^{\mathpzc{T}}_n (k)$.
\end{proof}

\begin{lemma}
	Given a schedule for a task-set $\mathpzc{T}$ of $n$ independent tasks with an \textit{occupied interval} $OI^{\mathpzc{T}}_n (k)$ for job $k$ of task $\tau_n$, it is always possible for a new task-set $\mathpzc{T} \cup \tau_{n+1}$ to schedule $\tau_{n+1}$ in such a way that a job $\iota_{n+1, k\prime}$ occurs within an interval $[t_1,t_2)$, where there is an amount of time $C_{n+1}$ available for $\tau_{n+1}$ and $t_2$ is equal to the start time of $OI^{\mathpzc{T}}_n (k)$.
\end{lemma}

\begin{proof}
	Since $U^{\mathpzc{T} \cup \tau_{i+1}}<1$ or $U^{\mathpzc{T} \cup \tau_{i+1}} = 1$ and the lcm of the periods exists, it is always possible to find a contingent interval $[t_i,t_2)$ before $OI^{\mathpzc{T}}_n (k)$ where there is an amount of time $C_{i+1}$ available for the execution of $\tau_{i+1}$. The only reason for which a job $\iota_{i+1, k\prime}$ of $\tau_{i+1}$ may not be able to execute within $[t_1,t_2)$ is that it could experience some blocking by previous jobs delaying the start time of $\iota_{i+1, k\prime}$. However, it is possible to change the phase $\phi_{i+1}$ in order to move the activations of the jobs of $\tau_{i+1}$ at an earlier moment in time, and therefore allow enough space before $[t_1,t_2)$ for the execution of the jobs previous to $\iota_{i+1, k\prime}$ that may delay its start time. Since the number of jobs of $\tau_{i+1}$ that may delay the start time of $\iota_{i+1, k\prime}$ is bounded by $wl_{i+1}-1$, it is always possible to find enough space before $[t_1,t_2)$ for their execution.
\end{proof}

\begin{lemma}
	Let all tasks of a set $\mathpzc{T}$ be strictly periodic and let $U^{\mathpzc{T}}<1$ or $U^{\mathpzc{T}}=1$ and the lcm of the periods exists. The best-case response time $BR_i$ of a task $\tau_i \in \mathpzc{T}$ is not influenced by its lower priority tasks.
\end{lemma}

\begin{proof}
	Assume that we find a schedule for a subset of task-set $\mathpzc{T}$ defined as $\mathpzc{T}_i = \{\tau_a | a \leq i, \tau_a \in \mathpzc{T}\}$ where the \textit{best-case response time} of $\tau_i$ is assumed and it occurs in an occupied interval $OI^{\mathpzc{T}_i}_i(k^{bcrt})$. Note that task-set $\mathpzc{T}_i$ contains the same tasks as $\mathpzc{T}$ without the lower priority tasks of $\tau_i$. Given Lemma 1 and Lemma 2, we can construct a schedule for a task-set $\mathpzc{T}_{i+1}$ where task $\tau_{i+1}$ is scheduled in such a way that a job $\iota_{i+1, k\prime}$ occurs in a contingent time interval $[t_1, t_2)$ before $OI^{\mathpzc{T}_i}_i(k^{bcrt})$. Hence, the jobs in the $OI^{\mathpzc{T}_i}_i(k^{bcrt})$ interval will remain unchanged and the \textit{best-case response time} $BR_i$ of $\tau_i$ will remain the same in $\mathpzc{T}_{i+1}$. Furthermore, note that since $\iota_{i+1, k\prime}$ occurs in a contingent time interval before the interval $OI^{\mathpzc{T}_i}_i(k^{bcrt})$, these two time intervals are contained in the longer occupied interval $OI^{\mathpzc{T}_{i+1}}_{i+1}(k\prime)$; hence, $\iota_{i, k^{bcrt}}$ occurs in $OI^{\mathpzc{T}_{i+1}}_{i+1}(k\prime)$ as well. Based on this observation and using Lemma 1 and Lemma 2 again, we can find a schedule for a task-set $\mathpzc{T}_{i+2}$ where the jobs in $OI^{\mathpzc{T}_{i+1}}_{i+1}(k\prime)$ remain unchanged and, therefore, $BR_i$ is preserved. Since we can continue constructing schedules preserving the \textit{best-case response time} of $\tau_i$ until $\mathpzc{T}_{i+l} = \mathpzc{T}$, we conclude that the \textit{best-case response time} of a task $\tau_i$ is not affected by its lower priority tasks.
\end{proof}

\section{Higher priority tasks}

\begin{lemma}
	Let $\mathpzc{T}$ be a set of strictly periodic tasks with a task $\tau_i \in \mathpzc{T}$. The threshold of higher priority tasks do not influence the response time of a job of $\tau_i$.
\end{lemma}

\begin{proof}
	In order to prove Lemma 4, first note that the response time $R_{i,k}$ of a job $\iota_{i,k}$ of $\tau_i$ is equal to $S_{i,k}+H_{i,k}$. Therefore, it is sufficient to show that the thresholds of higher priority tasks do not have an influence on the start time $S_{i,k}$ and the hold time $H_{i,k}$.
	
	\{\textit{Influence on} $S_{i,k}$\}. Under FPTS, a necessary condition for a job $\iota_{i,k}$ to start at a time $t$ after its activation is that the priority of $\tau_i$ is the highest among all tasks with pending load at time $t$. Since this condition only depends on the priority of the tasks, higher priority tasks will always affect the start time $S_{i,k}$ of $\iota_{i,k}$ independently of their thresholds.
	
	\{\textit{Influence on} $H_{i,k}$\}. Under FPTS, a job $\iota_{i,k}$ of a task $\tau_i$ can only be preempted by a higher priority task $\tau_h$ iff $\pi_h > \theta_i$. Since this condition is independent of the threshold $\theta_h$ of $\tau_h$, we conclude that the thresholds of higher priority tasks of $\tau_i$ do not have an influence on the hold time $H_{i,k}$ of $\iota_{i,k}$.
\end{proof}

\begin{thebibliography}{10}
	
	\bibitem{WS99}
	Y. Wang, and M. Saksena.
	Scheduling fixed-priority tasks with preemption thresholds.
	In Proc. 6th International Conference on Real-Time Computing Systems and Applications (RTCSA), December 1999.	
	
	\bibitem{BHKL12}
	R.J. Bril, M. van den Heuvel, U. Keskin, and J. Lukkien.
	Generalized fixed-priority scheduling with preemption thresholds.
	In Proc. 24th Euromicro Conference on Real-Time Systems (ECRTS), July 2012.
	
	\bibitem{BHL13}
	R.J. Bril, M.M.H.P. v.d. Heuvel, and J.J. Lukkien.
	Improved feasibility of fixed-priority scheduling with deferred preemption using preemption thresholds for preemption points.
	In Proc. 21st International Conference on Real-Time Networks and Systems (RTNS), ACM, pp. 225-264, October 2013.
	
	\bibitem{BFV08}
	R.J. Bril, G. Fohler, and W.F.J. Verhaegh. 
	Execution times and execution jitter of real-time tasks under fixed-priority preemptive scheduling. 
	Technical Report CSR 08-27, TU/e, The Netherlands, Oct. 2008. \url{http://www.win.tue.nl/~mholende/cantata/publications/BCG-WiP-ECRTS09-final.pdf}
	
	\bibitem{BLM13}
	R.J. Bril, J.J. Lukkien, and R.H. Mak.
	Best-case response times and jitter analysis of real-time tasks with arbitrary deadlines.
	In Proc. 21st International Conference on Real-Time Networks and Systems (RTNS), ACM, pp. 193-202, October 2013.
	
	\bibitem{C99}
	A. Cervin. 
	Improved Scheduling of Control Tasks.
	In Proc. 11th Euromicro Conference on Real-Time Systems (ECRTS), York, England, pp 4 - 10, 09 Jun 1999. 
	
	\bibitem{CRA99} 
	A. Crespo, I. Ripoll, and P. Albertos. 
	Reducing Delays in RT Control: The Control Action Interval. 
	In 14th IFAC World Congress on Automatic Control. Elsevier Science. p. 0. 1999.
	
	\bibitem{BC07} 
	G. Butazzo and A. Cervin. 
	Comparative Assessment and Evaluation of Jitter Control Methods. 
	In Proc. 15th International Conference on Real-Time and Network Systems (RTNS), Nancy, France, March 29-30, 2007.
	
	\bibitem{LCBRC06}
	M. Lluesma, A. Cervin, P. Balbastre, I. Ripoll, and A. Crespo.
	Jitter evaluation of real-time control systems.
	In Proceedings of the 12th IEEE International Conference on Embedded and Real-Time Computing Systems and Applications, 2006, pp. 257-260.
	
	\bibitem{THLT06}
	Y. Tian, Q. Han, D. Levy, and M. Tade. 
	Reducing Control Latency and Jitter in Real-Time Control.
	Asian Journal of Control, Vol. 8, No. 1, pp. 72-75, March 2006.
	
	\bibitem{BRVC04}
	P. Balbastre, I. Ripoll, J. Vidal, and A. Crespo.
	A task model to reduce control delays.
	Real-Time Syst., vol. 27, no. 3, pp. 215-236, Sep. 2004.
	
	\bibitem{BBGL99}
	S. Baruah, G. Buttazzo, S. Gorinsky, and G. Lipari. 
	Scheduling periodic task systems to minimize output jitter. 
	In Proc. International Conference on Real-Time Computing Systems and 
	Applications, pages 62-69, Hong Kong, December 1999.
	
	\bibitem{HHL10}
	S. Hong, X. Hu, and M. Lemmon.
	Reducing delay jitter of real-time control tasks through adaptive deadline adjustments.
	In Euromicro Conference on Real-Time Systems (ECRTS), 2010, pp. 229-238.
	
	\bibitem{MVFF01}
	P. Marti, R. Villa, J.M. Fuertes, G. Fohler. 
	On Real-Time Control Task Schedulability. 
	In European Control Conference (ECC), Porto, Portugal, 4-7 September 2001.
	
	\bibitem{MFFR01}
	P. Marti, J.M. Fuertes, G. Fohler, and K. Ramamritham. 
	Jitter Compensation for Real-Time Control System. 
	In Proceedings of the 22nd IEEE Real-Time Systems Symposium, pp 39-48, 2001.
	
	\bibitem{MFF01}
	P. Marti, J.M. Fuertes, G. Fohler. 
	Sampling Jitter Compensation in Real-time Control Systems. 
	In Proceedings of the 22nd IEEE Real-Time Systems Symposium (RTSS 2001), 
	London, UK, 2-6 December 2001.
	
	\bibitem{WB09}
	Y. Wu and M. Bertogna.
	Improving task responsiveness with limited preemptions.
	In Proc. 14th IEEE Int. Conf. Emerging Technol. Factory Autom. (ETFA 2009), 
	Mallorca, Spain, Sep. 2009.
	
	\bibitem{L90}
	J.P. Lehoczky.
	Fixed Priority Scheduling of Periodic Task Sets with Arbitrary Deadline.
	In Proceeding of the 11th IEEE Real-Time System Symposium (RTSS 1990), pp 201-209, December 1990.
	
	\bibitem{HKL91}
	M. G. Harbour, M. H. Klein, and J. P. Lehoczky. 
	Fixed priority scheduling periodic tasks with varying execution priority.
	Real-Time Systems Symposium, pages 116-128, December 1991.
	
	\bibitem{LL73}
	C. Liu and J. Layland.
	Scheduling algorithms for multiprogramming in a real-time environment.
	In Journal of the ACM, vol. 20, no. 1, pp. 46-61, January 1973.	
	
	\bibitem{KBL10}
	U. Keskin, R.J. Bril and J. Luikkien.
	Exact response-time analysis for fixed-priority preemption-threshold scheduling.
	In Proc. IEEE Conference on Emerging Technologies and Factory Automation (ETFA), Working-Progress Session, September 2010.
	
	\bibitem{BAHDB17}
	R.J. Bril, S. Altmeyer, M.M.H.P. van den Heuvel, R.I. Davis, and M. Behnam.
	Fixed priority scheduling with pre-emption thresholds and cache-related pre-emption delays: integrated analysis and evaluation.
	In Real-Time Systems, 31 January 2017.	
	doi:10.1007/s11241-016-9266-z
	
	\bibitem{GRASP}
	Mike Holenderski, Martijn M. H. P. van den Heuvel, Reinder J. Bril and Johan J. Lukkien.
	Grasp: Tracing, Visualizing and Measuring the Behavior of Real-Time Systems.
	In International Workshop on Analysis Tools and Methodologies for Embedded and Real-time Systems (WATERS), July 2010.
	
	
\end{thebibliography}
\end{document}

